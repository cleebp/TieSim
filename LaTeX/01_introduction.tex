\chapter{Introduction}

This undergraduate thesis presents both an overview of Virtual Reality (VR) as an educational tool, and also presents a rudimentary proof of concept for VR as a viable medium for competency based education. Throughout this thesis a discussion of both VR and its applications are presented, with special consideration towards the educational merit of the technology.

Chapter 2 focuses on the historical side of VR, and covers a concise overview of the technology since its conception in the early 1960s up until today. With a massive amount of development within VR in the past three years, Chapter 2 also goes into details about the current VR development trends.

Chapter 3 covers the historical and recent applications of VR for educational purposes. Over the years both the Department of Defense and NASA funded the technology and created a myriad of educational applications which are overviewed. As well, a look into the surgical applications of VR are discussed, as well as the potential STEM applications which are beginning to emerge.

Chapter 4 introduces and discusses at length the TieSim application, which as mentioned is meant to eventually serve as a proof of concept for VR as an educational tool. An overview of TieSim is presented, along with more detailed discussions of both the project's hardware and software. Of course development of TieSim was met with its fair share of obstacles which are examined, and finally the results of the project are outlined.

Chapter 5 outlines the proposed study for TieSim which would accompany a finished version of the simulation once development goals are met. The study itself is relatively simple and its procedures are listed, along with all of the more minute topics related to performing a study on human subjects such as recruitment, consent, risks, privacy, and data concerns.

Chapter 6 wraps up the research presented in this thesis, and proposes where future work could be focused, as well as summarizing the topics broadly covered throughout this work.

Finally, Appendix A presents the materials submitted to and accepted by the Institutional Review Board (IRB) at Appalachian State University. Since the study was not performed, these documents could be useful for anyone taking this project up in the future who wishes to obtain IRB approval for a similar study. One can use these documents to craft a similar proposal which would have a good chance at being accepted as these were.