\chapter{Conclusion}
\label{chap:conclusion}

Fundamentally, VR is an immersive technology which has yet to see a full proper consumer release. Before 2012, the majority of our society's interaction with VR was through often unrealistic glimpses at a futuristic technology that ultimately materialized in unfulfilling devices. Despite these setbacks VR has flourished throughout its history, and in the right settings it has proven to be a valuable and effective technology. Beyond the novelty of the technology, VR found a unique application in educational training simulations wherein the technology shined as both an effective and efficient tool. 

Currently we are experiencing an unprecedented moment in the history of VR development. Because of the dreams of one 19 year old geek from California, VR is now undergoing a renaissance of development, and there are now three major companies vying for a successful consumer product. Only time will tell which of these companies will stand the test of time, or whether they can exist harmoniously in the same product space. However, there is no doubt that VR will be the technology of our near and real future. In this future the technology will focus on the stereotypical usage case of video games; but, as was seen in the history of VR, educational outlets will arise. Projects like TieSim will ultimately pave the way for others to expound upon the concept of VR as a viable medium for competency based learning, and in a reality with multiple consumer HMDs, competition will drive these applications to break surpass all expectations. 

\section{Future Work}
\label{sec:futurework}

This thesis ultimate lays the groundwork for development of an educational VR simulation, TieSim, which can be used as a proof of concept for VR as a viable medium for competency based learning. Research has previously proven VR to be specifically applicable and effective towards certain skills, such as surgery (see section 3.1: Surgery). However, there is a lack of scholarly research on proving VR is applicable towards skills based learning in general. TieSim in its current state does not prove this notion, however, it does present the building blocks to do so. As an undergraduate thesis it is ambitious work that was ultimately halted due to a key hardware setback. 

The most obvious course for future work would be to take out the integration of the Leap Motion hardware which is currently halting future TieSim development (see section 4.5: Results and Conclusions). Once this hardware has been replaced it is not outside the scope of an additional undergraduate thesis to bring functionality of TieSim up to its original scope. Once development reaches this point, the simulation can finally be used as a proof of concept through a proposed study (see section 5.1: Procedures). 

%\section{Summary}
%\label{sec:summarystuff}
%
%Fundamentally, VR is an immersive technology which has yet to see a full proper consumer release. Before 2012, the majority of our society's interaction with VR was through often unrealistic glimpses at a futuristic technology, ultimately materialized in unfulfilling devices. Despite these setbacks VR has flourished throughout its history, and in the right settings it has proven to be a valuable and effective technology. Beyond the novelty of the technology, VR found a unique application in educational training simulations wherein the technology shined as both an effective and efficient tool. 
%
%Currently we are experiencing an unprecedented moment in the history of VR development. Because of the dreams of one 19 year old geek from California, VR is now undergoing a renaissance of development, and there are now three major companies vying for a successful consumer product. Only time will tell which of these companies will stand the test of time, or whether they can exist harmoniously in the same product space. However, there is no doubt that VR will be the technology of our near and real future. In this future the technology will focus on its stereotypical usage case of video games; but, as was seen in the history of VR educational outlets will arise. Projects like TieSim will ultimately pave the way for others to expound upon the concept of VR as a viable medium for competency based learning, and in a reality with multiple consumer HMDs competition will drive these applications to break the walls of anything one could anticipate. 